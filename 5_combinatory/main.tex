\section{Combinatória}

\subsection{Permutação}
    Dados um conjunto com n caixas iguais, queremos saber as diferentes formas de organizar n bolas.
    
    É descobrir todas as permutações por meio de uma função da STL de complexidade $O(n*n!)$
\begin{verbatim}
bool next_permutation (Iterator 1, Iterator 2);
//Rearranja os elementos na próxima permutação, sendo essa a menor 
possível maior que a atual (no quesito lexicográfico); retornando 
se a operação foi bem sucedida ou não.    
\end{verbatim}
    
    \subsubsection{Permutação sem repetição}
    \begin{equation}
        \centering
        P(n) = n!
    \end{equation}

    \subsubsection{Permutação com repetição}
    \begin{equation}
        \centering
        P(n, k, l, ...) = n!/k!l!...
    \end{equation}
    k, l,... representam a quantidade de vezes que a bola (k, l, ...) se repete

\subsection{Combinação}
    Dados um conjunto com n caixas iguais, queremos saber as possibilidades para se colocar k bolas iguais dentro delas\newline
    \begin{equation}
        \centering
        C(n, k) = \frac{n!}{k!(n-k)!}
    \end{equation}
        