\section{Template e Compilação}
\subsection{Template}
\begin{itemize}
    \item Template:
    \begin{verbatim}
        #include <bits/stdc++.h>
        using namespace std;

        #define endl '\n'
        #define ll long long
        #define ii pair<int,int>

        const int MAX = 212345;

        int main(){
            cin.tie(0);
            ios_base::sync_with_stdio(0);

            return 0;
        }
    \end{verbatim}
    \item Uma vez criado o template, para criar os arquivos da questão realizar o comando:
    \begin{verbatim}
        cp template.cpp X.cpp
    \end{verbatim}

\end{itemize}

\subsection{Compilação}
    \begin{enumerate}
        \item Entrar no diretório que está o arquivo .cpp
        \item Compilar (substituir "A.cpp" pelo arquivo .cpp em questão): 
        \begin{verbatim}
            g++ A.cpp -Wall -O2
        \end{verbatim}
        \item Executar o comando "./a.out $<$ tmp.in", onde tmp.in é o arquivo com as entradas\\
    \end{enumerate}
\pagebreak
%%%%%%%%%%%%%%%%%%%%%%%%%%%%%%%%%%%%%%%%%%%%%%%%%%%%%%%%%%%%%%%%%%%%%%%%%%%%%%%%%%%%%%%%%

